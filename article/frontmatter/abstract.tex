% DO NOT MOVE KEYWORDS BELOW ABSTRACT BLOCK!
% Otherwise, LaTeX will not render them (possible a bug in the template class file)
\keyword{Digital Twin}
\keyword{AI Agent Protocol}
\keyword{Semantic Interoperability}
\keyword{Digital Ecosystem}
\keyword{Cross-domain Collaboration}
\keyword{Smart City}
\keyword{Energy Grid}

\begin{abstract}

Digital Twins (DTs) have emerged as collaborative compositions of physical entities and their digital counterparts, 
    integrating real-time data streams, simulation models, and Artificial Intelligence (AI)-driven analytics across
    diverse domains including smart cities, energy grids, healthcare, and manufacturing.
These sophisticated systems enable real-time monitoring, predictive analytics, and optimization strategies.
They deliver substantial business value through
    reduced operational costs in manufacturing,
    improved energy efficiency in smart buildings, and
    enhanced predictive maintenance capabilities that prevent costly system failures.

However, as DT adoption grows and organizations recognize these benefits, the need for cross-domain collaboration creates significant interoperability challenges.
Current integration approaches face barriers including
    technical heterogeneity,
    semantic incompatibility between domain-specific ontologies,
    orchestration complexity, and
    high maintenance burdens that often require disruptive changes to operational systems.

While traditional approaches to DT interoperability typically rely on standardized protocols and unified data models,
    this research proposes a paradigm shift toward agent-mediated semantic interoperability.
By leveraging AI Agent Protocols within Digital Ecosystems, autonomous DT systems can 
    maintain their domain-specific optimizations while 
    dynamically negotiating semantic mappings for cross-domain collaboration.
This approach preserves the operational integrity of existing systems 
    while unlocking new possibilities for inter-domain coordination that were
    previously unfeasible due to integration complexity.

To achieve this concept,
    we propose a comprehensive exploration of AI Agent Protocols and their potential to transform DT ecosystems,
    followed by the design of a novel Digital Environment Architecture that formalizes how autonomous systems can cooperate without sacrificing their independence.
A proof-of-concept implementation extending open-source DT frameworks will demonstrate the practical viability of this approach, 
    evaluating through realistic cross-domain scenarios where Smart City and Energy Grid systems must collaborate to solve complex optimization challenges 
    that neither could address in isolation.
\end{abstract}
