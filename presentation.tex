\documentclass{beamer}

\usepackage[T1]{fontenc}
\usepackage[brazil]{babel}
\usepackage[utf8]{inputenc}

% Presentation strategy:
% Show previous presentation persona pains - OK
% Present research questions - OK
% Show small research keywords used
% Describe lack of AI-assisted DT development
% Highlight potential benefits of AI integration (e.g.: reduce costs, democratize access, etc)
% Present abstract
% Present research approach (MCP-based AI tools)
% - Explain why MCP
% - Explain how MCP can be used to create AI tools for DT development
% -- Explain MCP architecture
% -- Explain how MCP can be used to create AI agents and workflows
% - Present example tools to be developed
% -- CRD generator from ontology
% -- Tool configuration automation
% -- Cross-platform integration


% Choose the Inf theme
\usetheme{Inf}

% Define the title with \title[short title]{long title}
% Short title is optional
\title[Research study plan]
      {Research study plan}

% Optional subtitle
\subtitle{Research study plan for AI-assisted Digital Twin development}

\date{August 2025}

% Author information
\author{Henrique Krausburg Correa}
\institute{Institute of Informatics --- UFRGS}

\begin{document}

% Command to create title page
\InfTitlePage

\begin{frame}
  \frametitle{Agenda}
  \tableofcontents
\end{frame}

\section{Research question prospect}

\begin{frame}[plain]
 \sectionpage
\end{frame}

\frame{
    \frametitle{Case study}

    This research uses KTWIN \cite{Wermann_Wickboldt_2025} — a serverless Kubernetes-based Digital Twin platform — as a representative case study.
}

\frame{
    \frametitle{Case study}

    KTWIN workflow is based on three primary personas:

    \begin{itemize}
        \item DT Domain Expert
        \begin{itemize}
            \item Defines the Digital Twin model and its requirements.
            \item Responsible for the strategic direction and governance of the DT ecosystem.
            \item Verifies implementation.
        \end{itemize}
        \item DT Developer
        \begin{itemize}
            \item Implements the Digital Twin model using the KTWIN platform.
        \end{itemize}
        \item DT Operator
        \begin{itemize}
            \item Manages the deployment and operation of the Digital Twin in production environments.
            \item Creates CRD (Custom Resource Definition) files to model DT use case.
            \item Monitors system behavior through KTWIN observability solution (Grafana).
        \end{itemize}
    \end{itemize}
}

\frame{
    \frametitle{Case study}

    The following insights were gathered based on persona's tasks described in \cite{Wermann_Wickboldt_2025}:

    \begin{itemize}
        \item DT Domain expert is the most overloaded persona
        \begin{itemize}
            \item Responsible for both strategic modeling tasks and operational configuration duties.
            \item Faces challenges in managing the complexity of the Digital Twin ecosystem.
        \end{itemize}
        \item DT Developer is the “weakest link”
        \begin{itemize}
            \item Has few responsibilities compared to other personas.
        \end{itemize}
        \item DT Operator is often underestimated
        \begin{itemize}
            \item Could absorb DT domain expert / developer duties if empowered with an AI assistant.
        \end{itemize}
    \end{itemize}
}

\frame{
    \frametitle{Research questions}

    \begin{itemize}
        \item \textbf{RQ1:} How has artificial intelligence been used in Digital Twin development processes so far, and what gaps exist in current approaches?
        \item \textbf{RQ2:} How can artificial intelligence improve Digital Twin modeling, development, deployment, and operability through automated assistance and knowledge augmentation?
        \item \textbf{RQ3:} Can artificial intelligence contribute to solving interoperability issues in Digital Twin development through standardized integration frameworks and automated tool orchestration?
        \end{itemize}
}

\section{Literature review}

\begin{frame}[plain]
 \sectionpage
\end{frame}


\frame{
    \frametitle{Literature review: AI in Digital Twin development}

    To address \textbf{RQ1}, a preliminary literature review was conducted using Google Scholar with the following keywords:
    \begin{itemize}
        \item AI-Powered Digital Twin
        \item Digital Twin enabling technologies
        \item AI Digital Twin development
    \end{itemize}
}

\frame{
    \frametitle{Literature review: AI in Digital Twin development}

    The top 3 relevant papers (Google Scholar rank) were analyzed. Key findings include:
    \begin{itemize}
        \item Enabling technologies papers highlight the high complexity of Digital Twin development due to heterogeneous technologies and the vast number of required tools.
        \begin{itemize}
            \item \cite{Fuller_Fan_Day_Barlow_2020}, \cite{Hu_Zhang_Deng_Liu_Tan_2021}, \cite{Qi_Tao_Hu_Anwer_Liu_Wei_Wang_Nee_2021}.
        \end{itemize}
        \item Most AI applications are focused on enhancing production environments as part of the final Digital Twin solution.
         \begin{itemize}
            \item \cite{Alnaser_Maxi_Elmousalami_2024}, \cite{Sarp_Kuzlu_Jovanovic_Polat_Guler_2024}, \cite{Avanzato_Beritelli_Lombardo_Ricci_2024}
        \end{itemize}
        \item There is a lack of AI-powered tools or copilots designed to assist developers during the Digital Twin development process itself.
    \end{itemize}
}

\section{Abstract draft}

\begin{frame}[plain]
 \sectionpage
\end{frame}

\frame{
    \frametitle{Seções}

    \begin{itemize}
        \item O número a esquerda do cabeçalho marca a seção atual
        \item Logo a direita ele é seguido por:
        \begin{itemize}
            \item nome da seção
            \item subseção
            \item título do slide
        \end{itemize}
    \end{itemize}
}

\section{Enabling technologies}

\begin{frame}[plain]
 \sectionpage
\end{frame}

\frame{
    \frametitle{Teoremas}

    \begin{theorem}
        foo
    \end{theorem}
    \begin{proof}
        bar
    \end{proof}
}

\frame{
    \frametitle{Exemplos}

    \begin{exampleblock}{exemplo}
        foo
    \end{exampleblock}
}

\frame{
    \frametitle{Alertas}

    \begin{alertblock}{alerta}
        foo
    \end{alertblock}
}

\section*{}

\begin{frame}
    \frametitle{Obrigado!}
    \InfContacts
\end{frame}

\begin{frame}[allowframebreaks]
    \frametitle{Referências}
    \bibliographystyle{abntex2-alf}
    \bibliography{references}
\end{frame}

\end{document}
