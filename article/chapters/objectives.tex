\chapter{Research Goals and Expected Results}
\label{chap:objectives}

This research aims to address the critical challenge of achieving semantic interoperability between heterogeneous Digital Twin (DT) systems while preserving the autonomy and domain-specific optimizations of existing implementations.
The research question guiding this study is:

\textbf{How can AI Agent Protocols improve Digital Twin systems interoperability while preserving existing system autonomy?}

This question addresses the fundamental tension between the need for cross-domain DT collaboration and the practical constraints of maintaining operational systems that have been optimized for specific domains and use cases.

\section{Research Objectives}
\label{sec:objectives}

To answer the research question, this work pursues the following specific objectives:

\begin{enumerate}
    \item \textbf{Investigate AI Agent Protocols for DT Interoperability:} 
    Conduct a comprehensive review of AI Agent Protocols and analyze their applicability to Digital Twin ecosystems.
    This includes examining how agent-mediated communication can bridge semantic gaps between heterogeneous DT platforms
      without requiring universal standards or breaking changes.

    \item \textbf{Design a Digital Ecosystem Architecture:}
    Define and formalize an architectural framework for Digital Ecosystems powered by AI Agent Protocols.
      This architecture should enable autonomous DTs to cooperate through agent-mediated semantic mapping
        while maintaining their domain-specific characteristics and stakeholder independence.

    \item \textbf{Develop a Proof-of-Concept Implementation:}
    Create a working prototype that demonstrates cross-domain DT interoperability using open-source frameworks.
      The implementation will extend existing DT platforms with agent protocol capabilities to validate the proposed architectural approach.

    \item \textbf{Evaluate Cross-Domain Use Case Performance:}
    Conduct technical evaluation of the developed prototype using a realistic cross-domain scenario
      that requires coordination between heterogeneous DT systems to solve problems that individual systems cannot address independently.
\end{enumerate}

\section{Expected Contributions}
\label{sec:contributions}

This research is expected to make the following contributions to the Digital Twin and distributed systems communities:

\begin{itemize}
    \item \textbf{Conceptual Framework for Agent-Mediated DT Interoperability:}
    A novel theoretical framework that positions AI Agent Protocols as mediators for semantic interoperability in Digital Ecosystems,
      extending current understanding of how autonomous systems can cooperate without sacrificing domain specificity.

    \item \textbf{Digital Ecosystem Architecture:}
    A formalized architectural pattern that demonstrates how agent protocols can be integrated into existing DT platforms
      to enable cross-domain collaboration while maintaining system boundaries and stakeholder autonomy.

    \item \textbf{Open-Source Implementation:}
    A working prototype that extends open-source DT frameworks with AI Agent capabilities,
      providing a reference implementation for researchers and practitioners interested in exploring agent-mediated DT interoperability.

    \item \textbf{Cross-Domain Validation Scenario:}
    A comprehensive evaluation using Smart City and Energy Grid DTs that demonstrates practical benefits of the proposed approach in a realistic use case
      % involving traffic management and EV charging optimization.

    \item \textbf{Literature Review:} A comprehensive analysis of AI Agent Protocols in the context of Digital Twin systems,
      identifying gaps and opportunities for future research.
\end{itemize}

\section{Research Scope and Limitations}
\label{sec:scope-limitations}

This research focuses specifically on semantic interoperability challenges and does not address all aspects of DT integration. The scope includes:

\textbf{Included:} 
Semantic mapping between different ontologies,
  agent-mediated negotiation protocols, cross-domain optimization coordination, and
  open-source DT platform extensions.

\textbf{Excluded:}
Network-level protocols,
  security and authentication mechanisms,
  real-time synchronization guarantees, and
  proprietary DT platform integration.

The evaluation will be limited to simulated environments and may not capture all complexities of production DT deployments.
Results should be interpreted within the context of the specific use cases and platforms chosen for validation.
