\chapter{Methodology}
\label{chap:methodology}

This chapter describes the research methodology employed to investigate how AI Agent Protocols can enable semantic interoperability between
    heterogeneous Digital Twin systems.
The research follows a Design Science Research (DSR) approach, focusing on
    the design, development, and evaluation of a novel Digital Ecosystem architecture that leverages
    agent-mediated communication to solve cross-domain coordination challenges.

\section{Research Design}
\label{sec:research-design}

This research adopts the Design Science Research methodology proposed by Peffers et al.~\cite{peffers2007}, which provides a
    systematic framework for developing and evaluating Information Technology (IT) artifacts that address identified organizational problems.
DSR is particularly appropriate for this work because it emphasizes the creation and evaluation of purposeful artifacts---in this case,
    a Digital Ecosystem architecture and its proof-of-concept implementation.

The DSR process consists of six iterative activities:

\begin{enumerate}
    \item \textbf{Problem Identification and Motivation:}
    Addressed in Chapters~\ref{chap:introduction} and~\ref{chap:related-work}, identifying the semantic interoperability challenge in heterogeneous DT systems.
    
    \item \textbf{Objectives of a Solution:}
    Defined in Chapter~\ref{chap:objectives}, focusing on agent-mediated interoperability while preserving system autonomy.
    
    \item \textbf{Design and Development:}
    Described in Sections~\ref{sec:architecture-design} and~\ref{sec:implementation}, covering the Digital Ecosystem architecture and agent protocol integration.
    
    \item \textbf{Demonstration:}
    Presented in Section~\ref{sec:use-case}, using a realistic cross-domain scenario involving Smart City and Energy Grid DTs.
    
    \item \textbf{Evaluation:}
    Detailed in Section~\ref{sec:evaluation}, with metrics assessing semantic mapping capabilities, integration effort, and functional success.
    
    \item \textbf{Communication:}
    Fulfilled through this thesis and planned publications.
\end{enumerate}

This methodology enables rigorous artifact development while maintaining flexibility to iterate based on implementation findings and evaluation results.

\section{System Architecture Design}
\label{sec:architecture-design}

The proposed Digital Ecosystem architecture draws inspiration from the Destination Earth (DestinE) initiative's virtual cloud layer~\cite{Nativi_2021},
    adapting its principles for agent-mediated DT interoperability.
The architecture consists of three primary layers: DT platforms, mediation services, and orchestration.

\subsection{Virtual Cloud Architecture}
\label{subsec:virtual-cloud-arch}

Following DestinE's architectural principles, the virtual cloud layer virtualizes heterogeneous operational infrastructure environments while
    providing unified access through standardized interfaces.
For this research, a minimalist centralized orchestration approach has been selected to
    reduce implementation complexity while maintaining core capabilities for agent-mediated interoperability.

The virtual cloud architecture comprises:

\begin{itemize}
    \item \textbf{DT Platform Layer:}
    Independent KTWIN and Eclipse Ditto clusters, each maintaining domain-specific optimizations and ontologies.
    
    \item \textbf{Mediation Layer:}
    Agent-based services that broker semantic interactions between heterogeneous platforms using MCP for resource discovery and A2A for negotiation.
    
    \item \textbf{Orchestration Layer:}
    Centralized coordination services managing cross-domain requests and routing agent communications.
    
    \item \textbf{Service API Layer:}
    Common interfaces for external clients to submit cross-domain queries without platform-specific knowledge.
\end{itemize}

This layered approach isolates platform heterogeneity from
    cross-domain coordination logic, enabling agents to focus on semantic mediation rather than protocol translation.

\subsection{Digital Twin Platform Selection}
\label{subsec:platform-selection}

Two open-source DT platforms were selected to demonstrate heterogeneous system integration:

\textbf{KTWIN}~\cite{Wermann_Wickboldt_2025} serves as the Smart City DT platform.
KTWIN is a Kubernetes-based serverless framework designed to simplify DT adoption through cloud-native deployment patterns.
It natively supports the Digital Twins Definition Language (DTDL) and has been validated for smart city applications using the \texttt{opendigitaltwins-smartcities} ontology.
Key characteristics include:
\begin{itemize}
    \item Event-driven architecture using RabbitMQ message broker
    \item Kubernetes-native deployment for horizontal scalability
    \item Direct DTDL schema support without transformation
    \item Open-source implementation enabling protocol extensions
\end{itemize}

\textbf{Eclipse Ditto} serves as the Energy Grid DT platform.
Ditto is a mature open-source IoT platform providing DT capabilities through its ``Thing'' abstraction.
For this research, Ditto will implement the \texttt{opendigitaltwins-energygrid} ontology, which adapts the Common Information Model (CIM) standard to DTDL.
However, Ditto's native knowledge representation uses the Thing model rather than DTDL directly, requiring schema translation.
Key characteristics include:
\begin{itemize}
    \item Thing-based knowledge representation distinct from DTDL
    \item Rich REST and WebSocket APIs for twin interaction
    \item RabbitMQ integration (similar to KTWIN)
    \item Extensive documentation and active community
\end{itemize}

The deliberate selection of platforms with similar messaging infrastructure (RabbitMQ) but
    different knowledge representations (DTDL vs. Thing model) creates realistic heterogeneity while maintaining implementation feasibility.

\subsection{Agent Integration Architecture}
\label{subsec:agent-architecture}

Agents operate as autonomous mediators between DT platforms, bridging semantic gaps through dynamic negotiation rather than static mappings.
The agent architecture integrates two complementary protocols:

\textbf{Model Context Protocol (MCP)} enables resource discovery and schema introspection.
Each DT platform exposes an MCP server that publishes:
\begin{itemize}
    \item Available entity types and their properties
    \item Ontology structure and relationships
    \item Query capabilities and supported operations
    \item Real-time schema metadata for dynamic adaptation
\end{itemize}

\textbf{Agent-to-Agent (A2A) Protocol} facilitates semantic negotiation between platform agents.
When cross-domain coordination is required, agents use A2A to:
\begin{itemize}
    \item Exchange ontology concepts requiring mapping
    \item Propose semantic equivalences and transformations
    \item Negotiate acceptable mapping strategies
    \item Establish shared understanding for specific use cases
\end{itemize}

Each DT platform is paired with a dedicated agent that understands the platform's native knowledge representation.
The SmartCity Agent (paired with KTWIN) specializes in DTDL-based smart city ontologies, while
    the EnergyGrid Agent (paired with Ditto) specializes in CIM-based grid ontologies translated to the Thing model.
These agents act as semantic translators, exposing their platform's capabilities through MCP while negotiating cross-domain mappings through A2A.

A lightweight orchestration agent coordinates multi-platform interactions,
    routing requests to appropriate platform agents and aggregating responses.
This centralized coordination simplifies the proof-of-concept while maintaining the agent autonomy principle---platform agents retain
    full control over their semantic mappings and never directly access other platforms.

\section{Cross-Domain Use Case Definition}
\label{sec:use-case}

To validate the agent-mediated interoperability approach, a realistic cross-domain scenario has been designed that requires
    cooperation between Smart City and Energy Grid DT systems.

\subsection{Scenario Description: Dynamic EV Charging Coordination}
\label{subsec:scenario-description}

The use case addresses a practical urban energy management challenge:
    coordinating electric vehicle (EV) charging to prevent distribution grid overloads while maximizing charging service availability.

\textbf{Problem Context:}
During peak demand periods (e.g., 5-7 PM), large numbers of EVs require charging in urban areas.
Simultaneously, building loads and other consumers already stress the distribution network.
Uncoordinated EV charging can overload transformers, causing equipment damage, service interruptions, or cascading failures.

\textbf{Cross-Domain Requirement:}
Neither the Smart City DT nor the Energy Grid DT can solve this problem independently:
\begin{itemize}
    \item The Smart City DT knows EV locations, parking availability, and charging station infrastructure, but lacks knowledge of grid capacity constraints
    \item The Energy Grid DT monitors transformer loads and distribution network capacity, but has no visibility into EV movement or charging demand
\end{itemize}

\textbf{Scenario Execution:}
\begin{enumerate}
    \item 50 EVs enter downtown area between 5-7 PM, each requiring charging within 2 hours
    \item Multiple charging stations are available with varying capacities (7 kW, 22 kW, 50 kW fast chargers)
    \item Distribution transformers serving different downtown zones have different available capacities
    \item The system must assign EVs to charging stations that satisfy both:
        \begin{itemize}
            \item Parking availability and driver preferences (Smart City constraint)
            \item Transformer capacity limits (Energy Grid constraint)
        \end{itemize}
\end{enumerate}

\textbf{Success Criteria:}
\begin{itemize}
    \item All 50 EVs successfully charged within time window
    \item No transformer exceeds 95\% rated capacity
    \item System adapts if charging station status changes
    \item Cross-domain coordination achieved without manual integration code
\end{itemize}

\subsection{Semantic Gap Characterization}
\label{subsec:semantic-gaps}

The scenario requires bridging multiple semantic gaps between the
    \texttt{smartcities} ontology (KTWIN) and the
    \texttt{energygrid} ontology (Ditto).
These gaps are representative of real-world DT integration challenges:
    different domains model the same physical infrastructure from different perspectives using
    domain-specific ontologies optimized for their use cases.

\begin{table}[ht]
    \centering
    \small
    \begin{tabular}{ | m{4cm} | m{4cm} | m{5cm} | }
        \hline

        \textbf{Smart City Concept} & \textbf{Energy Grid Concept} & \textbf{Semantic Challenge} \\
        \hline

            \texttt{EVChargingStation} 

            (physical location) &
            \texttt{UsagePoint}

            (logical metering point) &
            Different abstraction levels: physical infrastructure vs. logical measurement point \\
        \hline

            \texttt{capacity}, \texttt{availableCapacity}
            
            (vehicles that can charge) &
            \texttt{ratedPower}, \texttt{estimatedLoad}
            
            (power capacity and consumption) &
            Different measurement units and semantics: vehicle count vs. electrical power metrics \\
        \hline

            \texttt{amperage}
        
            (electrical properties) &
            \texttt{ratedCurrent}
            
            (service ratings) &
            Similar concepts with different property names and potentially different semantic intent \\
        \hline

            \texttt{status} enum
            
            (working, outOfService, etc.) &
            \texttt{connectionState}, \texttt{amiBillingReady} & Operational state represented differently: service availability vs. connection/billing status \\
        \hline
    \end{tabular}
    \caption{Semantic gaps requiring agent-mediated mapping}
    \label{tab:semantic-gaps}
\end{table}

\subsection{Expected Agent Behaviors}
\label{subsec:agent-behaviors}

The agent-mediated coordination follows this workflow:

\textbf{Phase 1: Discovery (MCP)}
\begin{enumerate}
    \item External client submits request: ``Schedule charging for 50 EVs in downtown without exceeding transformer capacity''
    \item Orchestration agent invokes SmartCity Agent and EnergyGrid Agent
    \item SmartCity Agent exposes via MCP: ``I have 15 \texttt{ChargingStation} entities with properties: \texttt{location}, \texttt{maxPowerKW}, \texttt{currentOccupancy}''
    \item EnergyGrid Agent exposes via MCP: ``I have 3 \texttt{Transformer} entities with properties: \texttt{ratedCapacity}, \texttt{currentLoad}, \texttt{connectedUsagePoints}''
\end{enumerate}

\textbf{Phase 2: Semantic Negotiation (A2A)}
\begin{enumerate}
    \item EnergyGrid Agent: ``To calculate grid impact, I need to know which \texttt{UsagePoints} correspond to your \texttt{ChargingStations}''
    \item SmartCity Agent: ``My \texttt{ChargingStation\#5} has \texttt{maxPowerKW}=150 and \texttt{location}=(lat,lon). Can you map this to your grid topology?''
    \item EnergyGrid Agent performs spatial reasoning: ``\texttt{ChargingStation\#5} maps to my \texttt{UsagePoint\#UP\_789}, which connects to \texttt{Transformer\#T2}''
    \item Agents negotiate equivalence: \texttt{ChargingStation.maxPowerKW} $\approx$ \texttt{EquivalentLoad.activePower} (with opposite sign convention)
\end{enumerate}

\textbf{Phase 3: Collaborative Decision}
\begin{enumerate}
    \item For each charging station, agents jointly evaluate:
        \begin{itemize}
            \item SmartCity: parking availability, distance from EV locations
            \item EnergyGrid: connected transformer available capacity
        \end{itemize}
    \item EnergyGrid Agent: ``\texttt{Transformer\#T2} current load: 850kW, rated: 1000kW $\rightarrow$ available: 150kW''
    \item SmartCity Agent: ``Assigning 10 EVs to \texttt{ChargingStation\#5}''
    \item EnergyGrid Agent: ``\texttt{Transformer\#T1} at 95\% capacity $\rightarrow$ blocking \texttt{ChargingStation\#3}''
    \item SmartCity Agent: ``Acknowledged, redirecting EVs to alternative stations''
\end{enumerate}

This interaction demonstrates dynamic semantic mapping without pre-coordinated schemas or manual integration code.

\section{Implementation Methodology}
\label{sec:implementation}

The implementation follows an iterative agent-first development process, prioritizing early
    validation of the core agent protocol capabilities before full system integration.

\subsection{Development Phases}
\label{subsec:development-phases}

\textbf{Sprint 1: Agent Protocol Spike}

Focus: Validate MCP and A2A protocols work for semantic negotiation with minimal infrastructure.

\textit{Activities:}
\begin{itemize}
    \item Implement mock SmartCity Agent and EnergyGrid Agent with hardcoded schemas
    \item Create simplified scenario: 3 charging stations, 2 transformers
    \item Develop MCP server exposing mock entity schemas
    \item Implement A2A negotiation for basic semantic mappings
    \item Test agent discovery and mapping establishment
\end{itemize}

\textit{Deliverable:} Working agent communication demonstrating schema discovery and semantic mapping negotiation without full DT platforms.

\textit{Risk Mitigation:} Validating agent protocols early prevents wasted effort on DT infrastructure if protocols prove insufficient.

\textbf{Sprint 2: Single DT Platform Integration}

Focus: Integrate agents with real KTWIN deployment to validate platform introspection.

\textit{Activities:}
\begin{itemize}
    \item Deploy KTWIN cluster with \texttt{opendigitaltwins-smartcities} DTDL
    \item Create 3-5 \texttt{ChargingStation} digital twins with realistic properties
    \item Extend SmartCity Agent to introspect KTWIN's DTDL schemas via MCP
    \item Implement agent queries against live KTWIN twins
    \item Simulate EV charging requests with mock grid constraints
\end{itemize}

\textit{Deliverable:} SmartCity Agent dynamically querying KTWIN and recommending charging stations based on real twin data.

\textbf{Sprint 3: Heterogeneous Platform Integration}

Focus: Add Ditto platform and demonstrate cross-platform semantic bridging.

\textit{Activities:}
\begin{itemize}
    \item Deploy Eclipse Ditto cluster
    \item Translate \texttt{opendigitaltwins-energygrid} CIM-based DTDL to Ditto Thing model
    \item Create 2-3 \texttt{Transformer} twins with linked \texttt{UsagePoints}
    \item Implement EnergyGrid Agent with Ditto Thing API integration
    \item Develop spatial reasoning module mapping charging station locations to usage points
    \item Implement A2A semantic negotiation between SmartCity and EnergyGrid agents
\end{itemize}

\textit{Deliverable:} Agents successfully negotiate \texttt{ChargingStation} $\leftrightarrow$ \texttt{UsagePoint} mapping and answer cross-domain query:
    ``Which charging stations connect to Transformer T1?''

\textbf{Sprint 4: Complete Use Case and Orchestration}

Focus: Full scenario implementation with orchestration layer and evaluation.

\textit{Activities:}
\begin{itemize}
    \item Implement centralized orchestration agent
    \item Develop EV charging scheduler using agent-mediated coordination
    \item Create simulation framework with 50 EV agents and realistic load profiles
    \item Implement monitoring and metrics collection
    \item Execute scenario runs with varying EV arrival patterns and grid conditions
    \item Compare against baseline approaches (manual integration, shared ontology)
\end{itemize}

\textit{Deliverable:} Complete proof-of-concept demonstrating cross-domain optimization with quantitative evaluation results.

\subsection{Technology Stack}
\label{subsec:technology-stack}

The implementation leverages open-source technologies aligned with cloud-native practices:

\textbf{Infrastructure:}
\begin{itemize}
    \item \textbf{Kubernetes:} Container orchestration for KTWIN and Ditto clusters
    \item \textbf{Docker:} Containerization of all services for reproducibility
    \item \textbf{Terraform:} Infrastructure-as-code for cluster provisioning
    \item \textbf{RabbitMQ:} Message broker for event-driven DT communications
\end{itemize}

\textbf{Digital Twin Platforms:}
\begin{itemize}
    \item \textbf{KTWIN:} Smart City DT platform (native DTDL support)
    \item \textbf{Eclipse Ditto:} Energy Grid DT platform (Thing model)
\end{itemize}

\textbf{Agent Implementation:}
\begin{itemize}
    \item \textbf{Python 3.11+:} Primary language for agent development
    \item \textbf{MCP SDK:} Model Context Protocol client/server libraries
    \item \textbf{A2A Protocol:} Agent-to-Agent communication framework
    \item \textbf{Large Language Model:} For semantic reasoning and mapping (Claude Sonnet 3.5 or local Llama 3 model)
\end{itemize}

\textbf{Data and Ontologies:}
\begin{itemize}
    \item \textbf{opendigitaltwins-smartcities:} DTDL ontology for urban infrastructure
    \item \textbf{opendigitaltwins-energygrid:} CIM-based DTDL ontology for grid assets
    \item \textbf{Synthetic datasets:} EV mobility patterns, grid load profiles (IEEE test feeders)
\end{itemize}

\textbf{Monitoring and Evaluation:}
\begin{itemize}
    \item \textbf{Prometheus + Grafana:} Metrics collection and visualization
    \item \textbf{Custom evaluation framework:} Semantic mapping accuracy assessment
\end{itemize}

\subsection{Simulation Environment}
\label{subsec:simulation-environment}

To enable reproducible evaluation without requiring physical infrastructure, a simulation environment will be developed:

\textbf{EV Mobility Simulation:}
\begin{itemize}
    \item 50 EV agents with stochastic arrival times (normal distribution centered at 5 PM)
    \item Battery state-of-charge (SoC) modeling: 20-80\% range, charging rate dependent on station type
    \item Movement patterns based on synthetic urban mobility data
    \item Driver preferences: charging time constraints, station proximity
\end{itemize}

\textbf{Grid Load Simulation:}
\begin{itemize}
    \item Baseline transformer loads following typical urban consumption profiles
    \item Realistic load curves with peak demand 4-8 PM
    \item Three transformers with capacities: 1000 kW, 800 kW, 1200 kW
    \item Load forecasting uncertainty: ±10\% prediction error
\end{itemize}

\textbf{Charging Station Infrastructure:}
\begin{itemize}
    \item 15 charging stations distributed across 3 transformer service areas
    \item Mix of charging types: 7 kW (Level 2), 22 kW (Level 2 fast), 50 kW (DC fast)
    \item Dynamic availability status based on current EV occupancy
    \item Geographic distribution representative of urban downtown area
\end{itemize}

Time-series event replay enables deterministic scenario execution for controlled evaluation while
    introducing configurable variability to test agent adaptability.

\section{Evaluation Methodology}
\label{sec:evaluation}

The evaluation assesses whether agent-mediated interoperability achieves the claimed benefits over
    traditional integration approaches across three dimensions:
    semantic capability, integration effort, and functional performance. 
Table~\ref{tab:evaluation-metrics} covers these dimensions through several quantitative metrics.
These metrics collectively provide evidence for evaluating 
    DT interoperability improved by AI agent protocols while preserving system autonomy.

\begin{table}[H]
    \centering
    \small
    \begin{tabular}{|p{5cm}|p{5cm}|p{3cm}|}
        \hline
        
            \textbf{Metric} & \textbf{Measurement Method} & \textbf{Target Value} \\
        \hline
        
            \multicolumn{3}{|c|}{\textit{Semantic Interoperability}} \\
        \hline
        
            Semantic mapping coverage & \% of required concept mappings successfully established & >90\% \\
        \hline
        
            Zero-configuration integration & Number of manual semantic mappings required & 0 (fully automated) \\
        \hline
        
            Schema evolution tolerance & System adapts when new property added to ChargingStation? & Yes (no code changes) \\
        \hline
        
            Mapping negotiation time & Time for agents to establish initial mappings & <30 seconds \\
        \hline
        
            \multicolumn{3}{|c|}{\textit{Integration Effort}} \\
        \hline
        
            Implementation complexity & Lines of integration code (excluding agent framework) & <500 LOC \\
        \hline
        
            Configuration burden & Number of configuration files/mappings & <5 config files \\
        \hline
        
            Time to extend & Estimated hours to add 3rd DT platform (Weather) & <8 hours \\
        \hline
        
            \multicolumn{3}{|c|}{\textit{Functional Effectiveness}} \\
        \hline
        
            Cross-domain query success & \% of queries returning correct combined results & >95\% \\
        \hline
        
            Charging task completion & \% of 50 EVs successfully charged & 100\% \\
        \hline
        
            Grid constraint satisfaction & \% of time transformers remain below 95\% capacity & 100\% \\
        \hline
        
            Response time & End-to-end latency for charging assignment request & <5 seconds \\
        \hline
    \end{tabular}
    \caption{Evaluation metrics and targets}
    \label{tab:evaluation-metrics}
\end{table}

\subsection{Baseline Comparison}
\label{subsec:baseline-comparison}

To demonstrate the value of agent-mediated interoperability, two baseline approaches will be implemented for comparison:

\textbf{Baseline 1: Manual REST API Integration}

A traditional point-to-point integration approach:
\begin{itemize}
    \item Custom Python code queries KTWIN REST API for charging station data
    \item Separate custom code queries Ditto REST API for transformer data
    \item Hardcoded mapping logic translates between ontologies
    \item Static configuration file defines ChargingStation $\rightarrow$ UsagePoint relationships
\end{itemize}

\textit{Measurement focus:} Lines of code, configuration complexity, time to implement, brittleness to schema changes.

\textbf{Baseline 2: Shared Ontology Approach}

A unified semantic layer approach:
\begin{itemize}
    \item Design common ``EnergyInfrastructure'' ontology spanning both domains
    \item Implement exporters forcing KTWIN and Ditto to translate to shared schema
    \item Centralized query engine operates on unified model
\end{itemize}

\textit{Measurement focus:} Ontology development effort, deployment complexity, operational overhead, evolution challenges.

\textbf{Comparative Analysis:}

\begin{table}[ht]
    \centering
    \small
    \begin{tabular}{|l|c|c|c|}
        \hline
        
            \textbf{Evaluation Aspect} & \textbf{Manual API} & \textbf{Shared Ontology} & \textbf{Agent-Mediated} \\
        \hline
        
            Initial implementation time & Baseline & +50\% & +20\% \\
        
            Lines of integration code & Baseline & +30\% & -70\% \\
        
            Schema change impact & High & Medium & Low \\
        
            3rd system integration time & Baseline & +40\% & -60\% \\
        
            Runtime overhead & Low & Medium & Medium \\
        
            Semantic flexibility & None & Low & High \\
        \hline
    \end{tabular}
    \caption{Expected comparative results (to be validated)}
    \label{tab:baseline-comparison}
\end{table}

\subsection{Validation Approach}
\label{subsec:validation-approach}

The proof-of-concept will be validated through:

\textbf{Functional Validation:}
\begin{enumerate}
    \item \textbf{Nominal Scenario:} Execute base case with 50 EVs, verify all constraints satisfied
    \item \textbf{Stress Test:} Increase to 100 EVs, verify graceful degradation (some EVs delayed, no overloads)
    \item \textbf{Schema Evolution:} Add new property to ChargingStation DTDL, verify agents adapt without code changes
    \item \textbf{Failure Handling:} Simulate charging station outage, verify agents reroute EVs
\end{enumerate}

\textbf{Semantic Mapping Validation:}
\begin{itemize}
    \item Manual inspection of agent-negotiated mappings for correctness
    \item Sample validation: Execute 20 cross-domain queries, verify accuracy against ground truth
    \item Ontology expert review of mapping quality
\end{itemize}

\textbf{Performance Validation:}
\begin{itemize}
    \item Measure end-to-end latency for 100 requests
    \item Monitor resource utilization (CPU, memory, network) during peak load
    \item Compare against baseline approaches for equivalent workload
\end{itemize}

\textbf{Qualitative Assessment:}
\begin{itemize}
    \item Developer experience: subjective assessment of implementation effort
    \item Maintainability: estimated effort to update mappings vs. baseline
    \item Extensibility: feasibility analysis for adding Weather DT
\end{itemize}
