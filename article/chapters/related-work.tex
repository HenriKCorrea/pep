\chapter{Related work}
\label{chap:related-work}

A scoping review was conducted following the PRISMA-ScR reporting guidelines~\cite{tricco2018}
    to identify gaps in the current landscape regarding DT interoperability.
This review scope has been reduced to focus on gap identification of 8-12 highly relevant studies
    rather than comprehensive literature coverage.
This allows to synthesize evidence and assess the scope of literature on a topic while ensuring
    auditability and transparency.
The review process followed these steps:

\begin{enumerate}
    \item Define research question
    \item Selection of Digital Libraries
    \item Inclusion and exclusion criteria
    \item Primary study selection
    \item Data extraction and analysis
\end{enumerate}

\section{Research questions}

The following research questions build the foundation for this work, focusing on
    cross-domain DT interoperability solutions powered by AI agents:

\begin{enumerate}
    \item[RQ1:] What are the recent advances for Digital Twin interoperability?
    \item[RQ2:] Can Digital Twin interoperability be addressed by AI agents?
    \item[RQ3:] What are the open challenges for
        AI Agent protocols to handle semantic interoperability in order to solve
        cross-domain use cases between heterogeneous Digital Twin systems?
\end{enumerate}

\section{Search process}

The search process consisted in an automatic search of specific journal papers since 2020 in 
    Digital Libraries (DLs) presented on Table~\ref{tab:digital-library}.
Several constraints were defined for DL selection, such as
    open access to articles through institutional authentication.
Also, the DL must provide advanced search with boolean operator support for 
    title, abstract and keywords fields. 

\begin{table}[H]
    \centering
    \caption{Selected Digital Libraries}
    \label{tab:digital-library}
    \begin{tabular}{c | l | l}
        \textbf{ID} & \textbf{Digital Library} & \textbf{URL} \\
        \hline
        IEEE & IEEE Xplore   & \url{https://ieeexplore.ieee.org} \\
        SC   & ScienceDirect & \url{https://www.sciencedirect.com} \\
        SL   & Springer Link & \url{https://link.springer.com} \\
    \end{tabular}
\end{table}

To delimit the scope within research questions boundaries, three keyword sets have been defined.
The search query attempts to find works in which the authors explicitly cover DT, AI and interoperability.
Additionally, the query shall be adapted to each DL search syntax and field constraints.

\begin{lstlisting}[language=SQL, label=lst:search-query, caption=Search Query Structure, basicstyle=\ttfamily\small, breaklines, breakatwhitespace]
("digital twin") AND 
("interoperability" OR "cross-domain" OR "integrated platform") AND 
("agent protocol" OR "ai agent" OR "autonomous agent" OR "multi-agent" OR "agent-mediated" OR "agent communication" OR "large language model" OR "llm" OR "generative ai")
\end{lstlisting}

\section{Inclusion criteria}

English articles published since Jan 1 2020 that comply with
    at least one of following criteria must be included:

\begin{itemize}
    \item Studies about semantic interoperability or cross-domain integration between heterogeneous DT systems
    \item Studies applying AI agents, multi-agent systems, or autonomous agents to Digital Twin system
    \item Research proposing agent-based protocols or frameworks designed for heterogeneous DT systems
\end{itemize}

\section{Exclusion criteria}

Any non-English paper or publication of the following types must be excluded:

\begin{itemize}
    \item Duplicated publications
    \item Work focused on single-domain DT without addressing interoperability or cross-domain integration
    \item Articles without clearly stated research objectives, methodology or contribution
    \item Papers without full-text access
    \item Secondary studies (reviews, surveys)
    \item Non-peer-reviewed (preprints, working papers)
\end{itemize}

\section{Primary study selection process}

This review was designed to be performed by a single reviewer due to this work scope and time constraints.
First, the reviewer conducts the automated search. The Listing~\ref{lst:search-query} shall be applied.
Afterwards, the abstract, introduction and conclusion chapters are assessed
    based on inclusion and exclusion criteria.

Target primary study is at least 8 and at most 12 works among DLs.
If screening outcomes fewer than articles 8, expand to extra DLs interactively until
    achieving minimum coverage.
In contrast, if more than 12 papers are included,
    the selection will be reduced to 12 studies through prioritization based on:
    (1) Research questions coverage;
    (2) Publication recency; and
    (3) Citation count.
This approach is acceptable since formal quality assessment is not
    required in scope review methodology.

\section{Data extraction and analysis}

In the full-text review process, concepts related to DTs, AI and interoperability have been extracted.
Studies were tailored to our research questions following PRISMA-ScR guidelines for data charting.
For the comparative analysis (Table~\ref{tab:related-work-comparison}), the AI column specifically evaluates 
    whether AI agents or multi-agent systems are employed to address DT interoperability challenges,
    rather than merely being present in the DT system for other purposes (e.g., analytics, optimization).
This study presents both detailed summaries of each study and a 
    structured comparison (Table~\ref{tab:related-work-comparison}).

\section{Results}


%% Paragraph per paper
%%% - What DT interoperability issue the paper address?
%%% - Is AI agent technology being applied to solve the DT interoperability issue?
%%% - Is the validation performed on a cross-domain use case between heterogeneous Digital Twin systems?

Di Orio \textit{et al.} introduce an AI-enhanced Process Digital Twins (PDTs) framework, specifically tailored to support
    circular manufacturing practices, including reducing waste and extending product lifecycle~\cite{diorio2025a}.
PDT architecture is built on multi-viewpoint modeling approaches and adopts the Asset Administration Shell (AAS) paradigm,
    aligning with the Reference Architecture Model for Industry 4.0 (RAMI 4.0).
This enables flexible representation and interoperability in heterogeneous industrial settings.
PDT serves as the central integration hub, orchestrating data exchange through the AAS standardized communication.
Notably, all systems interacting with PDT like other DT's, external systems and applications, must adhere to PDT Cognitive AAS concept ---
    an AAS extension that encapsulates digital representation of process assets and embedded AI Models.
These models provide the means to PDT perform advanced analytics.
The PDT has been validated across three distinct industrial pilots ---
    WEEE (Waste Electrical and Electronic Equipment),
    BATT (Battery disassembly and second life evaluation), and
    PETRO (predictive maintenance in petrochemical processes).
Despite PDT ability to integrate multiple pilots within the manufacturing domain, 
    interoperability relies on standardization (all participants must adopt AAS).
%Autonomous agent protocols for cross-domain semantic negotiation remain unaddressed.

\cite{morabito2025a} present a DT framework for mobile network domain, aiming to improve
    connectivity and resource utilization.
The DT system is contextualized as a digital representation of the network defined as \emph{Campus Area Network},
    comprising heterogeneous devices (e.g.: smartphones) and network access technologies (e.g.: cellular 4G, WiFi and Bluetooth).
Data is exchanged between subsystems through Message Queuing Telemetry Transport (MQTT).
The framework is tested in an open working space at the University of Helsinki.
Although LLMs(Large Language Models) are discussed as a opportunity to enhance interoperability by generating data schemas,
    their work only uses AI to process analytics within DT framework to perform descriptive, diagnostic, predictive and prescriptive analysis.

\cite{hurtado2025a} implements the Self-configurable Manufacturing Industrial Agents (SMIA), a unified framework that
    enables the automation of DTs that act as the functional and social representatives of their corresponding assets.
Each agent is represented by a DT implemented as an industrial agent, self-configured based on the AAS of the
    asset it represents, interpreting its functionality through the Capability-Skill-Service (CSS) model,
    and exposing its capabilities as services available to the rest of the system.
A CSS-based ontology is created by using (OWL) ontology,
    adding semantic link to any element of the AAS model.
SMIA uses SPADE (Smart Python Agent Development Environment) as MAS execution platform,
    relying on the use of FIPA-ACL (Foundation for Intelligent Physical Agents)-(Agent Communication Language)
    for inter-agent communication.
Following the recommendations of FIPA for describing the behavior of agents,
    SPADE provides finite state machine (FSM) behavior.
The approach was validated in a simulated robotic logistics scenario, with the use case focused on
    assessing the viability of the self-configuration process.
Finally, authors recognize the need for AI agents, which could share or inherit agent capabilities,
    enabling the execution of manufacturing plans.

\cite{juarez2025a} proposes a semantic, modular solution for DT orchestration in industrial environments. 
The architecture is designed to provide
    real-time coordination, semantic interoperability and adaptive decision-making.
Interoperability is achieved through the orchestration layer that defines a
    Multi-Agent-System (MAS) with semantic APIs (Application Programming Interface) following
    the NGSI-LD standard.
The layer is composed by a management agent and recommendation agent.
Incoming data carries semantic context used by the agents to decide which task from a optimization layer shall be triggered and
    how to route the flow of execution.
The optimization layer hosts four independent AI agents to fulfill tasks such as 
    (a)~anomaly detection,
    (b)~behavioral classification,
    (c)~pattern optimization and
    (d)~threshold monitoring. 
A simulation has been carried out producing synthetic data for machining, assembly and inspection tasks.
In spite of semantic interoperability contributions, validation in a cross-domain environment presents an opportunity for future work.
Moreover, AI techniques are applied for analytics purposes, whereas the potential of AI agents in the orchestration layer remains an open research direction.

\cite{karapanagiotis2025a} proposes a Sliding Work Sharing (SWS) ontology built upon the Industrial Ontologies Foundry (IOF) 
    to enable semantic interoperability and reasoning across diverse domains.
The ontology formalizes concepts for Digital Twins, Human Digital Twins, and dynamic task flows 
    to support human-AI teaming in cyber-physical systems.
AI assistants leverage the ontology to provide contextual guidance to human operators, 
    such as virtual troubleshooting recommendations in manufacturing and task allocation suggestions in construction.
The approach is validated through cross-domain use cases involving collaborative robots and legacy systems.
While the shared ontology facilitates semantic integration, 
    it requires design-time alignment and does not address runtime semantic negotiation 
    between heterogeneous DT systems without shared vocabularies.
Nevertheless, SWS primary focus are human-AI interactions rather than autonomous DT-to-DT agent protocols.

%% Prisma ScR Items
%%
%% PRISMA-ScR 20 Results of Individual Sources of Evidence - For each included source of evidence, present the relevant data that were charted that relate to the review questions and objectives.
%% PRISMA-ScR 24 Summary of Evidence - Summarize the main results (including an overview of concepts, themes, and types of evidence available), link to the review questions and objectives, and consider the relevance to key groups.
%% PRISMA-ScR 26 Conclusions - Provide a general interpretation of the results with respect to the review questions and objectives, as well as potential implications or next steps.

\begin{figure}[H]
    \caption{PRISMA 2020 Flow Diagram}
    \label{fig:prisma}
    \resizebox{\textwidth}{!}{
    \prismaflowstart
        % Identification Phase
        % Individual database nodes
        \prismaflownode{n1}{}{Records identified from
            
        IEEE (n = 42)
        
        SD (n = 13)
        
        SL (n = ?)}{};

        \prismaflownode{n2}{below=of n1}{Records identified (n = 55)}{n1};
        
        % Screening Phase
        \prismaflownode{n3}{below=of n2}{Records screened (n = 55)}{n2};
        \prismaflownode{n3r}{right=of n3}{Records excluded (n = 50)

        - No DT inter. focus (n = 34)

        - Secondary study (n = 11)
        
        - No AI or Agent focus (n = 4)

        - No peer review (n = 1)
        }{};
        \prismaflowarrow{n3}{n3r};

        % Inclusion Phase
        \prismaflownode{n4}{below=of n3}{Studies included in qualitative synthesis (n = 5)}{n3};

        % Labels
        \prismalabel{1.3*\mh}{n1.west |- {$(n1)!-0.0!(n4)$}}{Identification};
        \prismalabel{1.3*\mh}{n1.west |- {$(n1)!0.5!(n4)$}}{Screening};
        \prismalabel{1.3*\mh}{n1.west |- {$(n1)!1.0!(n4)$}}{Included};

    \prismaflowend
    }
\end{figure}

\begin{table}[H]
    \centering
    \caption{Comparative Analysis of Related Work}
    \label{tab:related-work-comparison}
    \resizebox{\textwidth}{!}{
    \begin{tabular}{l|c|c|c|p{2.5cm}|p{2.2cm}|p{6.5cm}}
        \textbf{Reference} & \textbf{DT} & \textbf{AI} & \textbf{XD} & \textbf{Protocol/ Framework} & \textbf{Validation} & \textbf{Gap/Difference from This Research} \\
        \hline\hline
        
        \cite{morabito2025a} & Y & N & N & MQTT & Campus Area Network & 
            GenAI for analytics;
            lacks autonomous agent protocols for cross-domain DT collaboration \\
        \hline
        
        \cite{diorio2025a} & Y & N & N & AAS + RAMI 4.0 & 3 manufacturing pilots & 
            Multi-pilot integration within manufacturing domain;
            interoperability requires AAS standardization;
            no autonomous agent protocols for cross-sector semantic negotiation \\
        \hline
        
        \cite{juarez2025a} & Y & Y & N & NGSI-LD + MAS & Synthetic data & 
            Semantic orchestration within manufacturing domain;
            lacks cross-domain agent negotiation protocols \\
        \hline

        \cite{hurtado2025a} & Y & N & N & AAS + CSS + OWL + FIPA & Robotic logistics & 
            FIPA agents within unified standard stack; 
            design-time semantic configuration; 
            no autonomous negotiation for heterogeneous ecosystems across domains \\
        \hline
        
        \cite{karapanagiotis2025a} & Y & N & Y & IOF + SWS & Construction + Mfg & 
            Shared ontology (IOF) required;
            human-AI teaming;
            lacks autonomous DT-to-DT agent protocols \\
        \hline
        
        \textbf{This work} & \textbf{Y} & \textbf{Y} & \textbf{Y} & \textbf{AI Agent Protocols} & \textbf{Smart City $\leftrightarrow$ Energy Grid} & 
        \textbf{Autonomous AI Agent Protocols for cross-domain semantic interoperability in Digital Ecosystems without shared ontology} \\
        
    \end{tabular}
    }
\end{table}

\vspace{0.5em}
\noindent\textbf{Legend:}
    DT~=~Digital~Twin~Interoperability;
    AI~=~AI Agents/Multi-Agent~Systems;
    XD~=~Cross-Domain;
    Y~=~Yes;
    N~=~No;
    AAS~=~Asset~Administration~Shell;
    KG~=~Knowledge~Graph;
    MAS~=~Multi-Agent~System;
    Mfg~=~Manufacturing;
    GenAI~=~Generative~AI;
    IOF~=~Industrial~Ontologies~Foundry;
    RAG~=~Retrieval-Augmented~Generation;
    LLM~=~Large~Language~Model
