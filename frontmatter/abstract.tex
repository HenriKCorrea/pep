% This is the abstract for this research study plan.
% It should be one page long, containing the most relevant information about the intended contribution of the dissertation.

\begin{abstract}
Digital Twins have emerged as transformative technology for creating virtual replicas of physical systems, enabling real-time monitoring, simulation, and optimization across various industries. However, their development remains complex and resource-intensive, requiring significant expertise across multiple technological domains and creating substantial barriers to adoption, particularly for small and medium-sized enterprises. Current Digital Twin development processes suffer from persona overload, tool integration complexity, and limited application of artificial intelligence during the development lifecycle—with existing AI research focusing primarily on operational solutions rather than development assistance.

This research investigates how Artificial Intelligence can enhance productivity throughout the Digital Twin development process, addressing critical gaps in current development methodologies. Using KTWIN—a serverless Kubernetes-based Digital Twin platform—as a representative case study, this work aims to develop generalizable AI-assisted methodologies that democratize Digital Twin development while maintaining solution sophistication.

The proposed approach leverages the Model Context Protocol (MCP) specification to create specialized AI tools and connectors that automate Digital Twin development tasks, facilitate cross-platform integration, and bridge knowledge gaps between heterogeneous toolchains. The research addresses three primary questions: how AI has been applied to Digital Twin development processes, how AI can improve modeling and deployment workflows, and whether AI can solve interoperability challenges through standardized integration frameworks.

Expected contributions include a comprehensive analysis of current AI applications in Digital Twin development, a novel MCP-based framework for AI-assisted development processes, and practical tools demonstrating measurable productivity improvements. The research aims to lower adoption barriers, enabling organizations of all sizes to leverage Digital Twin technology while accelerating development cycles and reducing the complexity burden on development teams. Validation will be conducted through comparative development studies, tool integration metrics, and case study analysis using smart city scenarios.

This work positions itself at the intersection of DevOps engineering, Digital Twin technologies, and AI-assisted software development, contributing to multiple research communities while addressing real-world industry challenges in Digital Twin adoption and development efficiency.
\end{abstract}

\keyword{Digital Twins}
\keyword{Artificial Intelligence}
\keyword{Model Context Protocol}
\keyword{AI-Assisted Development}
\keyword{Serverless Computing}
\keyword{Kubernetes}
\keyword{Development Productivity}
\keyword{Tool Interoperability}