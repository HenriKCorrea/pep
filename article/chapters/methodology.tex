\chapter{Methodology}
\label{chap:methodology}

This chapter describes the research methodology employed to
    investigate how AgenTwin leverages AI Agent Protocols to enable semantic interoperability between
    heterogeneous Digital Twin systems.
The research follows a Design Science Research (DSR) approach, focusing on
    the design, development, and evaluation of AgenTwin's Digital Ecosystem architecture that
    enables agent-mediated communication to solve cross-domain coordination challenges.

\section{Research Design}
\label{sec:research-design}

This research adopts the Design Science Research methodology proposed by
    Peffers et al.~\cite{peffers2007}, which provides a
    systematic framework for developing and evaluating Information Technology (IT) artifacts that
    address identified organizational problems.
DSR is particularly appropriate for this work because it
    emphasizes the creation and evaluation of
    purposeful artifacts---in this case,
    AgenTwin's Digital Ecosystem architecture and its proof-of-concept implementation.

The DSR process consists of six iterative activities:

\begin{enumerate}
    \item \textbf{Problem Identification and Motivation:}
    Addressed in Chapters~\ref{chap:introduction} and~\ref{chap:related-work},
        identifying the semantic interoperability challenge in heterogeneous DT systems.
    
    \item \textbf{Objectives of a Solution:}
    Defined in Section~\ref{sec:objectives},
        focusing on agent-mediated interoperability while preserving system autonomy.
    
    \item \textbf{Design and Development:}
    Described in Sections~\ref{sec:architecture-design} and~\ref{sec:implementation-methodology},
        covering AgenTwin's Digital Ecosystem architecture and agent protocol integration.
    
    \item \textbf{Demonstration:}
    Presented in Section~\ref{sec:use-case},
        using a realistic cross-domain scenario involving Smart City and Energy Grid DTs.

    \item \textbf{Evaluation:}
    Detailed in Section~\ref{sec:evaluation-methodology}, assessing
        semantic capability,
        functional correctness and
        performance characteristics.
    
    \item \textbf{Communication:}
    Fulfilled through this thesis.
\end{enumerate}

\section{System Architecture Design}
\label{sec:architecture-design}

AgenTwin's Digital Ecosystem architecture
    depicted in Figure~\ref{fig:architecture} adopts a
    layered design inspired by
    DestinE virtual cloud layer concept~\cite{Nativi_2021}, 
    adapting its separation of concerns principles to agent-mediated Digital Twin interoperability.
Moreover, this OSI-inspired layering~\cite{day1983} provides clear separation between
    user interaction (Application),
    protocol translation (Presentation),
    conversation management (Session) and
    resource access (Domain, Integration, Platform).

\begin{figure}[htbp]
    \centering
    \includegraphics[width=\textwidth,keepaspectratio]{figures/architecture.pdf}
    \caption{
        AgenTwin architecture.
        The layered design integrates group chat patterns with MCP-based resource discovery
            to enable dynamic semantic negotiation between heterogeneous DT platforms.
    }
    \label{fig:architecture}
\end{figure}

\subsection{Application Layer}
\label{subsec:application-layer}

The application layer defines the end user that interacts with the Digital Ecosystem,
    which can be a human operator or an automated service.
A complex goal that requires assistance from multiple DT platforms is
    elaborated by the end user, then submitted to the system via agent.
Such goals may involve cross-domain queries, data aggregation, or coordinated actions
    that exceed the capabilities of any single DT platform.
Moreover, the application layer abstracts away platform-specific details,
    allowing users to focus on high-level objectives without needing to understand
    the underlying DT implementations.

\subsection{Presentation Layer}
\label{subsec:presentation-layer}

The presentation layer translates user intents into structured requests
    that can be processed by the Digital Ecosystem.
Specifically, the layer shall provide an application, service or AI agent that
    acts on behalf of the user
    to communicate with the Digital Ecosystem's agent through the A2A protocol.
This segregation allows the user application to remain agnostic of
    the underlying agent communication mechanisms,
    granting modularity and easier maintenance.
Furthermore to this point, the presentation layer may be implemented either as
    part of the Digital Ecosystem,
    integrated into the user application itself or
    provided as a third-party service,
    depending on the specific use case and system requirements.

\subsection{Session Layer}
\label{subsec:session-layer}

The session layer orchestrates multi-agent conversations that coordinate
    cross-domain interactions between heterogeneous DT platforms and user agents.
Unlike traditional orchestration approaches that rely on predetermined workflows,
    this layer enables dynamic task solving where the exact interaction sequence
    emerges from ongoing agent conversations rather than following pre-defined patterns.
This design is particularly suited for complex scenarios involving semantic interoperability,
    where the optimal coordination strategy cannot be determined a priori due to
    heterogeneous ontologies and unpredictable cross-domain dependencies.

The \textbf{Group Chat Manager} component realizes these principles through
    AutoGen's group chat pattern~\cite{wu2024autogen}, adapted for Digital Twin interoperability.
In this pattern, participating agents share the same conversation context and
    interact dynamically without strict communication order,
    enabling collaborative problem-solving that adapts to the specific characteristics
    of each cross-domain request.

The Group Chat Manager operates through three iterative steps.
First, it \textbf{dynamically selects a speaker} from available domain agents
    based on conversation context and role alignment.
This selection uses role-play prompts that consider both the current state of
    the cross-domain request and each agent's domain expertise,
    ensuring that the most appropriate agent contributes at each conversation turn.
Second, it \textbf{collects responses} from the selected agent,
    which may include semantic mappings, data queries, or requests for clarification
    from other domain agents.
Third, it \textbf{broadcasts messages} to all participating agents,
    maintaining shared context and enabling any agent to respond when their expertise is needed.

Resources are exclusively accessed through domain layer agents,
    which encapsulate platform-specific logic and knowledge.
The Group Chat Manager never directly interacts with DT platforms or their data sources;
    instead, it facilitates A2A protocol conversations among domain agents,
    which themselves use the Model Context Protocol (MCP) to
    query their respective platforms.
This protocol layering ensures clean separation between
    inter-agent coordination (A2A, session layer) and
    platform resource access (MCP, domain/integration layers).

Such design preserves platform autonomy,
    as domain agents mediate all interactions without external entities
    accessing their platforms' internals.
Digital Environment capabilities are exposed only through agent interfaces,
    enabling semantic mappings to be negotiated dynamically based on each request's context
    rather than requiring pre-coordinated schemas.
Aligned with DestinE's principles~\cite{Nativi_2021}, the session layer ensures that
    each DT platform retains full autonomy over its internal operations while
    participating in collaborative workflows mediated by agents.

Although not in this research scope, the session layer may enforce governance policies.
This includes ensuring that all interactions comply with
    security, privacy, and operational constraints
    defined by the Digital Environment administrators.
Policy enforcement occurs at the Group Chat Manager boundary, enabling centralized control
    while keeping domain agents focused on their platform-specific responsibilities.

\subsection{Domain Layer}
\label{subsec:domain-layer}

The domain layer encapsulates platform-specific agents that
    mediate interactions between the session layer and individual Digital Twin platforms.
Each \textbf{Digital Twin Agent} act as a domain specialist for a specific DT platform,
    understanding its native knowledge representation, ontology, and query capabilities.
These agents serve as semantic translators,
    exposing their platform's resources through the Model Context Protocol (MCP)
    while negotiating cross-domain mappings through the Agent-to-Agent (A2A) protocol.
To achieve semantic interoperability, besides being \textbf{domain-aware},
    each Digital Twin Agent must be capable to reason about
    semantic equivalences with other domains.
By isolating platform heterogeneity within dedicated agents,
    the domain layer enables seamless integration of diverse DT systems
    without requiring direct interoperability between platforms themselves.

\subsection{Integration Layer}
\label{subsec:integration-layer}

The integration layer provides the technical bridge between domain agents and Digital Twin platforms,
    implemented through Model Context Protocol (MCP) servers.
Each MCP server exposes a platform's capabilities through a standardized interface,
    enabling agents to discover available resources and their schemas dynamically.
This layer abstracts away platform-specific APIs and data access patterns,
    allowing agents to interact with heterogeneous DT systems through a unified protocol.

\subsection{Platform Layer}
\label{subsec:platform-layer}

The platform layer consists of individual Digital Twin systems
    that manage domain-specific digital representations of physical assets.
Each DT platform maintains its own internal data sources,
    including physical sensors, simulation models, historical datasets,
    and any other resources that contribute to its knowledge base.
DT platforms retain full autonomy over their data management,
    ensuring data integrity, security, and availability according to their own policies.
Agents access platform resources exclusively through the integration layer's MCP servers,
    preserving platform independence while enabling cross-domain coordination.

\section{Implementation Methodology}
\label{sec:implementation-methodology}

The AgenTwin proof-of-concept implementation realizes the proposed architecture through
    a concrete deployment integrating heterogeneous Digital Twin platforms mediated by AI agents.
Figure~\ref{fig:poc-architecture} illustrates the implementation architecture,
    showing the integration of KTWIN and Eclipse Ditto platforms through agent-based coordination.

\begin{figure}[htbp]
    \centering
    \includegraphics[width=\textwidth]{figures/proof-of-concept.pdf}
    \caption{AgenTwin proof-of-concept implementation architecture}
    \label{fig:poc-architecture}
\end{figure}

\subsection{User Application Component}
\label{subsec:user-application-component}

The User Application Component provides the end-user interface through which
    requests are submitted to the Digital Ecosystem.
To ensure research reproducibility while avoiding proprietary constraints,
    we use \textbf{Open WebUI}~\cite{baek2025},
    an open-source self-hosted AI interface that enables flexible agent protocol integration.

Open WebUI serves as both the user interface and agent client,
    translating natural language user requests into structured agent communications.
The platform's extensibility allows implementation of custom Agent-to-Agent (A2A) protocol clients
    without vendor lock-in to proprietary chat interfaces or cloud services.

\subsection{Digital Ecosystem Orchestration}
\label{subsec:digital-ecosystem-orchestration}

The Digital Ecosystem's agent communication is realized through
    \textbf{AutoGen}~\cite{wu2024autogen},
    an open-source AI agent framework focused on multi-agent conversation.
Participating agents can
    share conversation context and
    interact without predetermined communication sequences.
This provides the foundation for implementing the
    Group Chat Manager component described in Section~\ref{subsec:session-layer}.

AutoGen's speaker selection mechanism can be configured with custom logic that
    prioritizes agents based on conversation context analysis,
    ensuring semantic negotiation progresses efficiently toward cross-domain resolution.
The LLM backend flexibility enables evaluation across both
    cloud-based models (E.g.: Claude Sonnet 3.5 via Anthropic API) and
    locally-hosted alternatives (E.g.: Llama 3 via Ollama).
This facilitates prototyping under realistic constraints while
    supporting performance comparison and cost-sensitive deployment scenarios.

\subsection{Digital Twin Platform Components}
\label{subsec:digital-twin-platform-components}

Two open-source DT platforms were selected to demonstrate heterogeneous system integration
    while maintaining realistic semantic complexity.
The selection criteria prioritized platforms with
    sufficient heterogeneity to require semantic mediation, and
    extensibility to support MCP server integration.

\textbf{KTWIN}~\cite{Wermann_Wickboldt_2025} implements the Smart City DT platform.
KTWIN is a Kubernetes-based serverless framework designed to 
    simplify DT adoption through cloud-native deployment patterns.
It natively supports the Digital Twins Definition Language (DTDL) and has been
    validated for smart city applications using the
    \texttt{opendigitaltwins-smartcities} ontology.
This ontology provides comprehensive models for
    urban infrastructure entities including
    parking facilities,
    EV charging stations,
    traffic sensors, and
    building management systems.

Key technical characteristics include:

\begin{itemize}
    \item \textbf{Serverless Event-Driven Architecture:}
    Function-as-a-Service (FaaS) deployment enables
        automatic scaling and
        reduced operational overhead, while
        the centralized event broker (RabbitMQ) supports
        real-time state updates through
        multiple protocols (MQTT, HTTP, AMQP).

    \item \textbf{Kubernetes-Native Infrastructure:}
    Custom Resource Definitions (CRDs) and operators automate
        infrastructure provisioning based on
        DTDL ontology definitions,
        enabling multi-cloud and edge deployments without vendor lock-in.
    
    \item \textbf{Ontology-Driven Automation:}
    Native DTDL support eliminates schema translation overhead,
        allowing direct implementation of the \texttt{smartcities} ontology concepts.
\end{itemize}

\textbf{Eclipse Ditto} implements the Energy Grid DT platform.
Ditto is a mature open-source IoT platform providing
    DT capabilities through its ``Thing'' abstraction,
    widely adopted in industrial IoT deployments.
For this research, Ditto manages twins conforming to the
    \texttt{opendigitaltwins-energygrid} ontology, which adapts the
    Common Information Model (CIM) standard---the energy industry's reference ontology---to DTDL.
Since Ditto's native knowledge representation uses the Thing model,
    schema translation will be necessary for this research.

Key technical characteristics include:

\begin{itemize}
    \item \textbf{Thing-Based Knowledge Representation:}
    Distinct from KTWIN's DTDL approach,
        necessitating genuine semantic mapping rather than trivial schema alignment
    
    \item \textbf{Rich REST and WebSocket APIs:}
    Well-documented interfaces simplify MCP server implementation for resource access
    
    \item \textbf{RabbitMQ Integration:}
    Similar messaging infrastructure to KTWIN reduces deployment complexity
        while maintaining semantic heterogeneity at the ontology level
    
    \item \textbf{Production-Grade Maturity:}
    Extensive documentation, active community, and industrial adoption
        ensure realistic platform behavior for evaluation
\end{itemize}

The deliberate selection of platforms with similar
    messaging infrastructure (RabbitMQ) but
    different knowledge representations (native DTDL vs. Thing model)
    creates the realistic heterogeneity necessary to
    validate agent-mediated semantic interoperability while
    maintaining implementation feasibility within research constraints.
Neither platform provides native MCP server capabilities.
Implementing MCP servers for both KTWIN and Eclipse Ditto
    constitutes part of this research scope.

\subsection{Agent-Platform Integration}
\label{subsec:agent-interaction-impl}

While Section~\ref{subsec:session-layer} described
    agent-to-agent coordination through the A2A protocol,
    this subsection focuses on the complementary agent-to-platform interaction pattern
    realized through the Model Context Protocol (MCP).
MCP enables dynamic resource discovery and query execution without requiring agents to
    maintain platform-specific client code or
    hardcoded schema knowledge.

Each DT platform exposes an MCP server.
The server publishes three types of information:
    resource catalogs (available entity types and instances),
    schema metadata (properties, relationships, semantic annotations), and
    query capabilities (supported operations like filtering, spatial queries, aggregation).
For KTWIN, the MCP server wraps Kubernetes CRDs and RabbitMQ event streams,
    exposing \texttt{smart-city} ontology entities.
For Ditto, the MCP server interfaces with REST and WebSocket APIs,
    querying \texttt{energy-grid} ontology entities.

Each domain agent embeds an MCP client
    that connects to its associated platform's MCP server.
The client provides the agent's LLM with tool definitions for
    discovering resources,
    retrieving schemas, and
    executing queries with filtering and projection.
This architecture enables agents to dynamically discover platform capabilities through MCP,
    negotiate semantic mappings through A2A conversations, and
    execute coordinated queries to solve cross-domain problems---all
    without requiring modifications to the underlying DT platforms beyond MCP server deployment.

\section{Cross-Domain Use Case Definition}
\label{sec:use-case}

To validate AgenTwin's agent-mediated interoperability approach, a realistic cross-domain scenario has been designed that requires
    cooperation between Smart City and Energy Grid DT systems.

\subsection{Scenario Description: Dynamic EV Charging Coordination}
\label{subsec:scenario-description}

The use case addresses a practical urban energy management challenge:
    coordinating electric vehicle (EV) charging to prevent distribution grid overloads while maximizing charging service availability.

\textbf{Problem Context:}
During peak demand periods (e.g., 5-7 PM), large numbers of EVs require charging in urban areas.
Simultaneously, building loads and other consumers already stress the distribution network.
Uncoordinated EV charging can overload transformers, causing equipment damage, service interruptions, or cascading failures.

\textbf{Cross-Domain Requirement:}
Neither the Smart City DT nor the Energy Grid DT can solve this problem independently:
\begin{itemize}
    \item The Smart City DT knows EV locations, parking availability, and charging station infrastructure, but lacks knowledge of grid capacity constraints
    \item The Energy Grid DT monitors transformer loads and distribution network capacity, but has no visibility into EV movement or charging demand
\end{itemize}

\textbf{Scenario Execution:}
\begin{enumerate}
    \item 50 EVs enter downtown area between 5-7 PM, each requiring charging within 2 hours
    \item Multiple charging stations are available with varying capacities (7 kW, 22 kW, 50 kW fast chargers)
    \item Distribution transformers serving different downtown zones have different available capacities
    \item The system must assign EVs to charging stations that satisfy both:
        \begin{itemize}
            \item Parking availability and driver preferences (Smart City constraint)
            \item Transformer capacity limits (Energy Grid constraint)
        \end{itemize}
\end{enumerate}

\textbf{Success Criteria:}
\begin{itemize}
    \item All 50 EVs successfully charged within time window
    \item No transformer exceeds 95\% rated capacity
    \item System adapts if charging station status changes
    \item Cross-domain coordination achieved without manual integration code
\end{itemize}

\subsection{Semantic Gap Characterization}
\label{subsec:semantic-gaps}

The scenario requires bridging multiple semantic gaps between the
    \texttt{smart-city} ontology (KTWIN) and the
    \texttt{energy-grid} ontology (Ditto).
These gaps are representative of real-world DT integration challenges:
    different domains model the same physical infrastructure from different perspectives using
    domain-specific ontologies optimized for their use cases.
Table~\ref{tab:semantic-gaps} summarizes the main semantic gaps identified.

\begin{table}[H]
    \centering
    \small
    \begin{tabular}{ | m{4cm} | m{4cm} | m{5cm} | }
        \hline

        \textbf{Smart City Concept} & \textbf{Energy Grid Concept} & \textbf{Semantic Challenge} \\
        \hline

            \texttt{EVChargingStation} 

            (physical location) &
            \texttt{UsagePoint}

            (logical metering point) &
            Different abstraction levels: physical infrastructure vs. logical measurement point \\
        \hline

            \texttt{capacity}, \texttt{availableCapacity}
            
            (vehicles that can charge) &
            \texttt{ratedPower}, \texttt{estimatedLoad}
            
            (power capacity and consumption) &
            Different measurement units and semantics: vehicle count vs. electrical power metrics \\
        \hline

            \texttt{amperage}
        
            (electrical properties) &
            \texttt{ratedCurrent}
            
            (service ratings) &
            Similar concepts with different property names and potentially different semantic intent \\
        \hline

            \texttt{status} enum
            
            (working, outOfService, etc.) &
            \texttt{connectionState}, \texttt{amiBillingReady} &
            Operational state represented differently: service availability vs. connection/billing status \\
        \hline
    \end{tabular}
    \caption{Semantic gaps requiring agent-mediated mapping}
    \label{tab:semantic-gaps}
\end{table}

\section{Evaluation Methodology}
\label{sec:evaluation-methodology}

The evaluation assesses whether AgenTwin's agent-mediated interoperability architecture
    successfully enables cross-domain coordination between heterogeneous Digital Twin systems.
This evaluation focuses on demonstrating functional feasibility and
    measuring key characteristics of AgenTwin within
    the constraints of a Master's thesis timeline.

The evaluation is organized into three complementary dimensions that
    collectively demonstrate the viability of agent-mediated semantic interoperability:
    semantic capability,
    functional correctness and
    performance characteristics.

\subsection{Semantic Mapping Evaluation}
\label{subsec:semantic-mapping-evaluation}

Semantic mapping evaluation assesses the quality of
    agent-negotiated correspondences between heterogeneous ontologies:

\begin{itemize}
    \item \textbf{Mapping Coverage:}
    Measure the percentage of required concept mappings
        (identified in Table~\ref{tab:semantic-gaps}) that
        agents successfully establish during cross-domain coordination.
    This quantifies the extent to which agents can autonomously bridge semantic gaps
        without manual intervention.

    \item \textbf{Mapping Correctness:}
    Manual inspection of agent-negotiated mappings to verify semantic accuracy.
    For critical mappings (e.g., \texttt{EVChargingStation} $\leftrightarrow$ \texttt{UsagePoint}),
        validate that agents correctly identify relationships and
        handle unit conversions (vehicle count vs. power metrics).

    \item \textbf{Cross-Domain Query Accuracy:}
    Execute a representative set of cross-domain queries with known expected outcomes.
    Since no external ground truth exists for this novel integration,
        correctness is validated through independent manual verification:
        inspecting source data in both platforms and
        confirming agent results match semantic relationships defined in Table~\ref{tab:semantic-gaps}.
    This demonstrates whether agent-mediated integration produces semantically correct results.
\end{itemize}

\subsection{Functional Evaluation}
\label{subsec:functional-evaluation}

Functional evaluation verifies that the system successfully coordinates
    cross-domain operations under various operational conditions:

\begin{itemize}
    \item \textbf{Nominal Scenario:}
    Execute the cross-domain use case described in Section~\ref{sec:use-case}.
    Verify that all constraints are satisfied:
        all EVs receive charging assignments,
        transformers remain within safe capacity limits, and
        assignments respect both Smart City availability and Energy Grid capacity constraints.
    This demonstrates the system's ability to perform basic cross-domain coordination.
    
    \item \textbf{Stress Test:}
    Increase demand beyond nominal capacity to test system behavior under resource contention.
    Observe whether the system exhibits graceful degradation:
        prioritizing assignments to prevent grid overloads while
        clearly communicating when some EVs must be delayed or redirected.
    This validates robustness under constrained resources.
    
    \item \textbf{Schema Evolution:}
    Add a new property (\texttt{reservationSupport}) to the ChargingStation DTDL model.
    Verify that agents adapt to the schema change without requiring code modifications,
        demonstrating the architecture's resilience to ontology evolution.
    This tests whether the system maintains zero-configuration integration when ontologies change.
\end{itemize}

\subsection{Performance Evaluation}
\label{subsec:performance-evaluation}

Performance evaluation characterizes the operational overhead of
    agent-mediated coordination to assess practical feasibility:

\begin{itemize}
    \item \textbf{Mapping Negotiation Time:}
    Measure the time required for agents to establish initial semantic mappings
        for the EV charging coordination scenario.
    This characterizes the startup overhead of agent-mediated interoperability
        compared to pre-configured integration approaches.
    
    \item \textbf{Query Response Time:}
    Measure end-to-end latency for charging assignment requests,
        from user submission to coordinated response.
    Report mean, median, and 95th percentile latencies across multiple executions
        to characterize typical and worst-case performance.
    This determines whether response times are acceptable for interactive use.
    
    \item \textbf{Resource Utilization:}
    Monitor CPU, memory, and network usage during peak load scenarios.
    Document resource consumption patterns to inform deployment requirements and
        assess feasibility for typical cloud or edge computing environments.
    This validates that the approach does not impose prohibitive infrastructure demands.
\end{itemize}
