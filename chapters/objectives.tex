% This chapter presents the objectives and expected results for this research study plan.
% It should be one to two pages long, clearly indicating the objectives of the work and the concrete final result intended.

\chapter{Research Goals and Expected Results}

\section{Research Questions}

This research aims to investigate and demonstrate how Artificial Intelligence can enhance productivity throughout the Digital Twin development process, with a focus on creating generalizable AI-assisted methodologies using KTWIN as a representative case study. The research is guided by three primary research questions:

\textbf{RQ1:} How has artificial intelligence been used in Digital Twin development processes so far, and what gaps exist in current approaches?

\textbf{RQ2:} How can artificial intelligence improve Digital Twin modeling, development, deployment, and operability through automated assistance and knowledge augmentation?

\textbf{RQ3:} Can artificial intelligence contribute to solving interoperability issues in Digital Twin development through standardized integration frameworks and automated tool orchestration?

\section{Research Approach and Innovation}

This research proposes a novel approach to AI-assisted Digital Twin development through the implementation of specialized tools based on the Model Context Protocol (MCP) specification. MCP provides a standardized framework for connecting AI models to diverse data sources and tools, enabling the creation of sophisticated AI agents and workflows that can interact with the complex Digital Twin development ecosystem.

The research innovation lies in developing MCP-based connectors and automation tools that can address specific Digital Twin development tasks—such as generating Kubernetes Custom Resource Definitions from ontology specifications, automating tool configuration, and facilitating cross-platform integration. This approach promises to democratize Digital Twin development by reducing the expertise barriers while maintaining the sophistication and capability of the resulting solutions.

By leveraging the recent open-sourcing of GitHub Copilot Chat extension and the growing ecosystem of AI development tools, this research positions itself at the intersection of three rapidly evolving fields: DevOps and platform engineering, Digital Twin technologies, and AI-assisted software development.

\section{Expected Contributions}

This research aims to deliver both theoretical and practical contributions to the Digital Twin development community. The expected outcomes include a comprehensive analysis of current AI applications in Digital Twin development, a novel framework for AI-assisted development processes, and practical tools that demonstrate measurable productivity improvements. The ultimate goal is to lower the barriers to Digital Twin adoption, enabling smaller organizations to leverage this transformative technology while accelerating development cycles for enterprises of all sizes.

The following chapters detail the methodology for achieving these objectives, including planned activities, validation strategies, and a timeline for completing this research program.