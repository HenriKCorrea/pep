\chapter{Introduction}
\label{chap:introduction}


Digital Twins (DTs) are virtual representations of physical entities that enable monitoring, simulation, and optimization of complex systems throughout their lifecycle.
By integrating data from sensors, IoT devices, and computational models, DTs provide actionable insights into the inner operations of systems and predict future behaviors in ways that traditional monitoring technologies cannot~\cite{Botín-Sanabria_2022}.
Recent advances in artificial intelligence (AI), machine learning (ML), big data analytics, and ubiquitous connectivity have accelerated DT adoption across diverse domains, including smart cities, energy grids, healthcare, manufacturing, and agriculture~\cite{Kshetri_2021}.

DT applications range from single-asset monitoring---such as Shell's structural DT for the Bonga oilfield's facility---to high-fidelity geographical representations like the European Commission's Destination Earth initiative~\cite{Kshetri_2021}.
These implementations demonstrate DTs' capacity to reduce operational costs, improve maintenance efficiency, and support decision-making under uncertainty.
However, as DT ecosystems expand, cross-domain use cases emerge that require cooperation between heterogeneous DT platforms.
For instance, a supply chain DT \cite{Gerlach_Zarnitz_2021} requires historical and real-time data from manufacturing DTs, warehousing DTs, and cargo DTs to optimize end-to-end operations.
As DT adoption grows, the need for interoperability between such heterogeneous platforms becomes increasingly critical to unlock cross-domain value and enable collaborative solutions that transcend the boundaries of individual systems.


Interoperability is the ability of different DT systems to seamlessly integrate and communicate despite using different standards, protocols, or ontologies.
Standards like Digital Twins Definition Language (DTDL)~\cite{opendigitaltwins_2025} for smart cities and the Common Information Model (CIM)~\cite{Uslar_2012} for energy grids are used to describe the structure and behavior of digital twins.
Despite the growing adoption of DTs, achieving interoperability between heterogeneous DT systems remains a significant challenge~\cite{Tripathi_2024}.
Current integration approaches face several barriers:

\begin{itemize}
    \item \textbf{Technical interoperability:} The diversity of proprietary platforms, data formats, and communication protocols used across domains creates integration challenges. Most industrial software applications are proprietary, preventing widespread adoption of open-source solutions and complicating system integration.
    
    \item \textbf{Semantic interoperability:} Different domains model their systems using domain-specific ontologies (e.g., DTDL for smart cities, CIM for energy grids) that require manual mapping and translation.

    \item \textbf{Orchestration complexity:} Multiple DTs must coordinate physical-to-physical, virtual-to-virtual, and virtual-to-physical interactions to solve complex cross-domain problems that can't be solved by individual systems.

    \item \textbf{Maintenance burden:} Existing integration solutions often rely on static, pre-coordinated schemas that break when systems evolve, placing significant cognitive load on domain experts to maintain integrations over time.
\end{itemize}

These challenges highlight the need for flexible, non-intrusive interoperability solutions that can adapt to the diversity of existing DT implementations without requiring stakeholders to make disruptive changes to operational systems or abandon domain-optimized tools.

Recent advances propose the concept of \emph{Digital Ecosystems} (DEs)---collaborative environments where autonomous DTs, stakeholders, and resources interact through open, common network services and APIs.
Nativi et al.~\cite{Nativi_2021} present the Destination Earth Digital Ecosystem, 
    which demonstrates how a virtual-cloud framework can provide interoperability between software components 
    by publishing a set of open, common, and consistent network services and related APIs.
This approach creates a paradox of autonomous yet cooperative entities by decoupling DTs from the underlying cloud infrastructure, 
    enabling stakeholders to maintain domain-specific optimizations while fostering collaboration when cross-domain value is required.

Building on this vision, this research explores the use of AI Agent Protocols
    to facilitate semantic interoperability between DTs within such digital ecosystems.
Rather than pursuing universal standards that require breaking changes to operational systems, agent-mediated negotiation and mapping can bridge the gap between diverse ontologies and data models.
This approach aligns with Klar's framework for DT maturity~\cite{Klar_2024},
    where semantic interoperability represents a critical advancement that enables DTs to cooperate meaningfully while preserving their domain-specific characteristics and stakeholder autonomy.

% This work extends the open-source KTWIN platform---a serverless Kubernetes-based DT framework that provides high-level abstraction without compromising scalability---with agent protocol capabilities to demonstrate cross-domain DT cooperation. The validation focuses on a Smart City and Energy Grid scenario, where KTWIN models traffic management and EV charging infrastructure using DTDL ontology for Smart Cities, while Eclipse Ditto manages energy grid operations using DTDL ontology for Energy Grids. Through MCP-enabled semantic interoperability, these heterogeneous systems can coordinate EV charging optimization strategies that neither could achieve independently, demonstrating how agent protocols advance DT maturity toward meaningful interoperability without requiring universal standards or breaking changes to existing systems.
