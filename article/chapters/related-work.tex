\chapter{Related work}
\label{chap:related-work}

To identify gaps in the current landscape regarding DT interoperability, we conducted a
    systematic literature review (SRL) following the guidelines proposed by~\cite{kitchenham2007}.
We adapted this review to focus on gap identification by incorporated elements from the
    PRISMA-ScR checklist~\cite{tricco2018}.
This allows a through and unbiased approach for literature summarization while ensuring
    repeatability, auditability and transparency.
The review process is defined by the following steps:

\begin{enumerate}
    \item Define research question
    \item Selection of Digital Libraries
    \item Inclusion and exclusion criteria
    \item Primary study selection
    \item Data extraction and analysis
\end{enumerate}

\section{Research questions}

The following research questions builds the foundation for this work, focusing on
    cross-domain DT interoperability solutions powered by AI agents:

\begin{enumerate}[label=RQ\arabic*:]
    \item \label{rq:digital-twin} What are the recent advances for Digital Twin interoperability?
    \item \label{rq:ai-agents} Have AI agents been applied to address Digital Twin interoperability challenges?
    \item \label{rq:cross-domain} How can AI Agent protocols be designed to 
        enable semantic interoperability to solve
        cross-domain use cases between heterogeneous Digital Twin systems?
\end{enumerate}

\section{Search process}

The search process is a automatic search of specific journal papers since 2020 in 
    Digital Libraries (DLs) presented on Table~\ref{tab:digital-library}.
Several constrains were defined for DL selection, such as
    open access to articles through institutional authentication.
Also, the DL must provide advanced search with boolean operator support for 
    author keywords field exclusively.

\begin{table}[H]
    \centering
    \caption{Selected Digital Libraries}
    \label{tab:digital-library}
    \begin{tabular}{c | l | l}
        \textbf{ID} & \textbf{Digital Library} & \textbf{URL} \\
        \hline
        IEEE & IEEE Xplore   & \url{https://ieeexplore.ieee.org} \\
        SC   & ScienceDirect & \url{https://www.sciencedirect.com} \\
        MDPI & MDPI          & \url{https://www.mdpi.com} \\
    \end{tabular}
\end{table}

To delimit the RSL scope within research questions boundaries, three keywords sets, restricted to authors, are defined below.
The search query attempts to find works which authors explicit covers DT, AI and interoperability.
Additionally, the query shall be adapted to each DL search syntax and field constraints.

\begin{lstlisting}[language=SQL, label=lst:search-query, caption=Search Query Structure, basicstyle=\ttfamily\small, breaklines, breakatwhitespace]
("digital twin") AND 
("interoperability" OR "cross-domain" OR "integrated platform") AND 
("agent protocol" OR "ai agent" OR "autonomous agent" OR "multi-agent" OR "agent-mediated" OR "agent communication" OR "large language model" OR "llm" OR "generative ai")
\end{lstlisting}

\section{Inclusion criteria}

English articles published since Jan 1 2020 that comply with
    at least one of following criteria must be included:

\begin{itemize}
    \item Studies about semantic interoperability or cross-domain integration between heterogeneous DT systems
    \item Works involving DTs with AI agents, multi-agent systems or autonomous agents
    \item Research proposing protocols or frameworks applicable for heterogeneous DT systems
\end{itemize}

\section{Exclusion criteria}

Any non-english paper of The following types must be excluded:

\begin{itemize}
    \item Pure theoretical papers without any empirical validation, simulation, or prototype implementation
    \item Duplicated publications
    \item Studies not addressing at least one research question
    \item Articles without clear objectives or outcomes
    \item Work focused on single-domain DT without addressing interoperability or cross-domain integration
\end{itemize}

\section{Primary study selection process}

This review was designed to be performed by a single reviewer.
First, the reviewer conducts the automated search. The Listing~\ref{lst:search-query} shall be applied.
Afterwards, the abstract, introduction and conclusion chapters are assessed
    based on inclusion and exclusion criteria.

\section{Data extraction and analysis}

In the full-text review process, concepts related to DTs, AI and interoperability have been extracted.
Studies were tailored to our research questions following PRISMA-ScR guidelines for data charting.
This study presents both a structured comparison (Table~\ref{tab:related-work-comparison}) and detailed summaries of 
each study.

\section{Results}

\begin{figure}[H]
    \caption{PRISMA 2020 Flow Diagram}
    \label{fig:prisma}
    \resizebox{\textwidth}{!}{
    \prismaflowstart
        % Identification Phase
        % Individual database nodes
        \prismaflownode{n1}{}{Records identified from
            
        IEEE (n = 1)
        
        SD (n = 13)
        
        MDPI (n = 1)}{};
        
        % Screening Phase
        \prismaflownode{n3}{below=of n1}{Records screened (n = 15)}{n1};
        \prismaflownode{n3r}{right=of n3}{Records excluded (n = ?)}{};
        \prismaflowarrow{n3}{n3r};

        % Eligibility Phase
        \prismaflownode{n4}{below=of n3}{Full-text articles assessed for eligibility (n = ?)}{n3};
        \prismaflownode{n4r}{right=of n4}{Full-text articles excluded \\ - Not Relevant (n = ?) \\ - Not Open/Available paper (n = ?)}{};
        \prismaflowarrow{n4}{n4r};

        % Inclusion Phase
        \prismaflownode{n5}{below=of n4}{Studies included in qualitative synthesis (n = ?)}{n4};

        % Labels
        \prismalabel{1.3*\mh}{n1.west}{Identification};
        \prismalabel{1.3*\mh}{n1.west |- {$(n1)!0.75!(n4)$}}{Screening};
        \prismalabel{1.3*\mh}{n1.west |- n5}{Included};

    \prismaflowend
    }
\end{figure}

\begin{table}[H]
    \centering
    \caption{Comparative Analysis of Related Work}
    \label{tab:related-work-comparison}
    \resizebox{\textwidth}{!}{
    \begin{tabular}{l|c|c|c|c|c|p{4cm}}
        \textbf{Reference} & \textbf{DT Interop.} & \textbf{AI Agents} & \textbf{Domains} & \textbf{Protocol / Framework} & \textbf{Validation} & \textbf{Gap/Difference} \\
        \hline
        Author (2023) & Yes & No & No & N/A & Prototype & Single-domain only \\
        % ... more rows
    \end{tabular}
}
\end{table}

